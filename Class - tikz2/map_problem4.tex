% Under Creative Commons Attribution licence 3.0
% (http://creativecommons.org/licences/by/3.0)
% Author: Florian Lesaint
\documentclass[landscape]{article}
\usepackage{tikz}
%%%<
\usepackage{verbatim}
\usepackage[active,tightpage]{preview}
\PreviewEnvironment{tikzpicture}
\setlength\PreviewBorder{10pt}%
%%%>
\begin{comment}
:Title: Cuboid in a 2 vanishing points perspective
:Tags: 3D;Geometry;Mathematics
:Author: Florian Lesaint
:Slug: cuboid

This code draws a cuboid using a 2 vanishing points perspective.
Within the code, parameters can be revised to tune the drawing.
\end{comment}
\usetikzlibrary{calc}
\begin{document}
\begin{tikzpicture}
	%%% Edit the following coordinate to change the shape of your
	%%% cuboid
      
	%% Vanishing points for perspective handling
	\coordinate (P1) at (-7cm,3cm); % left vanishing point (To pick)
	\coordinate (P2) at (8cm,4cm); % right vanishing point (To pick)

	%% (A1) and (A2) defines the 2 central points of the cuboid
	\coordinate (A1) at (0em,0cm); % central top point (To pick)
	\coordinate (A2) at (0em,-1cm); % central bottom point (To pick)

	%% (A3) to (A8) are computed given a unique parameter (or 2) .8
	% You can vary .8 from 0 to 1 to change perspective on left side
	\coordinate (A3) at ($(P1)!.8!(A2)$); % To pick for perspective 
	\coordinate (A4) at ($(P1)!.8!(A1)$);

	% You can vary .8 from 0 to 1 to change perspective on right side
	\coordinate (A7) at ($(P2)!.7!(A2)$);
	\coordinate (A8) at ($(P2)!.7!(A1)$);

	%% Automatically compute the last 2 points with intersections
	\coordinate (A5) at
	  (intersection cs: first line={(A8) -- (P1)},
			    second line={(A4) -- (P2)});
	\coordinate (A6) at
	  (intersection cs: first line={(A7) -- (P1)}, 
			    second line={(A3) -- (P2)});

	%%% Depending of what you want to display, you can comment/edit
	%%% the following lines

	%% Possibly draw back faces

%	\fill[gray!90] (A2) -- (A3) -- (A6) -- (A7) -- cycle; % face 6
%	\node at (barycentric cs:A2=1,A3=1,A6=1,A7=1) {\tiny f6};
%	
%	\fill[gray!50] (A3) -- (A4) -- (A5) -- (A6) -- cycle; % face 3
%	\node at (barycentric cs:A3=1,A4=1,A5=1,A6=1) {\tiny f3};
%	
%	\fill[gray!30] (A5) -- (A6) -- (A7) -- (A8) -- cycle; % face 4
%	\node at (barycentric cs:A5=1,A6=1,A7=1,A8=1) {\tiny f4};
%	
%	\draw[thick,dashed] (A5) -- (A6);
%	\draw[thick,dashed] (A3) -- (A6);
%	\draw[thick,dashed] (A7) -- (A6);

	%% Possibly draw front faces

	% \fill[orange] (A1) -- (A8) -- (A7) -- (A2) -- cycle; % face 1
	% \node at (barycentric cs:A1=1,A8=1,A7=1,A2=1) {\tiny f1};
%	\fill[gray!50,opacity=0.2] (A1) -- (A2) -- (A3) -- (A4) -- cycle; % f2
%	\node at (barycentric cs:A1=1,A2=1,A3=1,A4=1) {\tiny f2};
%	\fill[gray!90,opacity=0.2] (A1) -- (A4) -- (A5) -- (A8) -- cycle; % f5
%	\node at (barycentric cs:A1=1,A4=1,A5=1,A8=1) {\tiny f5};

	%% Possibly draw front lines
	\draw[] (A1) -- (A2);
	\draw[] (A3) -- (A4);
	\draw[] (A7) -- (A8);
	\draw[] (A1) -- (A4);
	\draw[] (A1) -- (A8);
	\draw[] (A2) -- (A3);
	\draw[] (A2) -- (A7);
	\draw[] (A4) -- (A5);
	\draw[] (A8) -- (A5);
	
	% Possibly draw points
	% (it can help you understand the cuboid structure)
%	\foreach \i in {1,2,...,8}
%	{
%	  \draw[fill=black] (A\i) circle (0.15em)
%	    node[above right] {\tiny \i};
%	}
	% \draw[fill=black] (P1) circle (0.1em) node[below] {\tiny p1};
	% \draw[fill=black] (P2) circle (0.1em) node[below] {\tiny p2};

\coordinate (B) at  ($(A1)!0.4!(A2)$);
\coordinate (Z) at  ($(A1)!0.6!(A2)$);
\coordinate (V) at  ($(A4)!0.4!(A3)$);
\coordinate (W) at  ($(A4)!0.6!(A3)$);

\draw [fill=gray, opacity=0.2]
       (B) -- (Z) -- (W) -- (V) -- cycle;

\coordinate (B2) at  ($(A1)!0.571!(A8)$);
\coordinate (B3) at  ($(A1)!0.857!(A8)$);

\draw [fill=gray, opacity=0.2]
       (B) -- (B2) -- (B3) -- (Z) -- cycle;

\coordinate (B4) at  ($(A4)!0.571!(A5)$);
\coordinate (B5) at  ($(A4)!0.857!(A5)$);

\draw [fill=gray, opacity=0.2]
       (B2) -- (B3) -- (B5) -- (B4) -- cycle;

\draw [] ([shift=(-90:0.5cm)]B5) -- ++(164:0.25 cm);
\draw [] ([shift=(-90:0.5cm)]B5) -- ++(-16:0.25 cm);
\draw [] ([shift=(-90:0.5cm)]B5) -- ++(-160:0.15 cm) node[below] {\tiny{$35^{\circ}$}};


\coordinate (C1) at  ($(B2)!0.2!(B4)$);
\coordinate (C2) at  ($(B3)!0.2!(B5)$);
\coordinate (C3) at  ($(B3)!0.75!(B5)$);

%\draw [dashed] (C1) node [below=-2pt] {\tiny{$b$}}-- (C2) node [above=-1pt] {\tiny{$a$}};
\draw [dashed] (C1) node [below=-2pt] {\tiny{$b$}}-- (C3) node [above=-1pt] {\tiny{$c$}};

\end{tikzpicture}
\end{document} 
